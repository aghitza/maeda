\documentclass[11pt]{article}

\usepackage[english]{babel}
\usepackage{csquotes}
\usepackage[style=alphabetic, 
            hyperref=auto,
            url=true,
            isbn=false,
            doi=false,
            block=none,
            maxnames=10,
            backend=biber]{biblatex}
\addbibresource{maeda.bib}
\renewcommand{\bibfont}{\normalfont\small}

\usepackage{amssymb}
\usepackage{amsthm}
\usepackage{amsmath}

\usepackage[pagewise, mathlines]{lineno}
\linenumbers
\modulolinenumbers[2]

\usepackage{gitinfo}
\usepackage[english]{isodate}

\usepackage{geometry}
\usepackage{numprint}

\theoremstyle{plain}
\newtheorem{theorem}{Theorem}[section]
\newtheorem{conjecture}[theorem]{Conjecture}
\newtheorem{proposition}[theorem]{Proposition}
\theoremstyle{definition}
\newtheorem{definition}[theorem]{Definition}
\theoremstyle{remark}
\newtheorem{remark}[theorem]{Remark}
\newtheorem{example}[theorem]{Example}

\numberwithin{equation}{section}
\numberwithin{table}{section}
\renewcommand{\labelenumi}{(\arabic{enumi})}
\newcommand{\longto}{\longrightarrow}

\newcommand{\CC}{\mathbb{C}}
\newcommand{\ZZ}{\mathbb{Z}}
\newcommand{\FF}{\mathbb{F}}
\newcommand{\QQ}{\mathbb{Q}}
\newcommand{\NN}{\mathbb{N}}
\newcommand{\cB}{\mathcal{B}}
\newcommand{\cH}{\mathcal{H}}
\renewcommand{\SS}{\mathfrak{S}}
\renewcommand{\Im}{\operatorname{Im}}

\newcommand{\bound}{\numprint{10000}}

\newcommand{\GL}{\mathrm{GL}}
\newcommand{\SL}{\mathrm{SL}}

\cleanlookdateon
\title{Experimental evidence for Maeda's conjecture on modular forms\footnote{Commit: \gitAbbrevHash{} on \printdate{\gitAuthorDate{}}}
}
\author{
Alexandru Ghitza\footnote{Research of the first author was supported by 
a Discovery Grant from the Australian Research Council.}  
{} and 
Angus McAndrew\\
Department of Mathematics and Statistics\\
University of Melbourne\\
{\tt aghitza@gmail.com}, {\tt mcandrew@student.unimelb.edu.au}
}
\date{}


\begin{document}
\thispagestyle{empty}

\maketitle
\begin{abstract}
  We describe a computational verification of Maeda's conjecture for the Hecke
  operator $T_2$ in all weights less than $\bound$.
\end{abstract}


\section{Introduction}

TODO: Motivation.

We review some basic definitions and properties of modular forms.  For
details, the reader is invited to consult~\cite{Stein}.

Let $k\in\ZZ$.  A \emph{modular form} of level $1$ and weight $k$ is a
holomorphic function
\begin{equation*}
  f\colon\cH\longto\CC, \quad\text{where }
  \cH=\{z\in\CC\mid \Im z>0\},
\end{equation*}
satisfying
\begin{itemize}
  \item Modularity: for all $z\in\cH$ and all
    $g=\left(\begin{smallmatrix}a&b\\c&d\end{smallmatrix}\right)\in\SL_2(\ZZ)$,
      \begin{equation*}
        f\left(\frac{az+b}{cz+d}\right)=(cz+d)^kf(z).
      \end{equation*}
  \item Holomorphicity at $i\infty$: a holomorphic function $f$ satisfying the
    modularity condition satisfies $f(z+1)=f(z)$ for all $z\in\cH$, so it has
    a Fourier expansion
    \begin{equation*}
      f(z)=\sum_{n=-\infty}^\infty a_nq^n,\quad\text{where we set }
      q=e^{2\pi i z}.
    \end{equation*}
    We ask for $f$ to be \emph{holomorphic at $i\infty$}, i.e. that $a_n=0$
    for all $n<0$.
\end{itemize}

We say that a modular form $f$ is a \emph{cusp form} if $a_0=0$.  The cusp
forms of weight $k$ form a vector space $S_k$.  These vector spaces are
equipped with a family of \emph{Hecke operators} $T_m$ (for $m\in\NN$), whose
effect on the Fourier expansion $f(q)=\sum a_nq^n$ of $f\in S_k$ is given by
\begin{equation*}
  (T_m f)(q)=\sum_{n=1}^\infty \left(\sum_{d\mid\gcd(m,n)}d^{k-1}a_{mn/d^2}\right)q^n.
\end{equation*}

The complex vector space $S_k$ has dimension 
\begin{equation*}
  d=\begin{cases}
    \left[\frac{k}{12}\right]-1 & \text{if }k\equiv 2\pmod{12},\\
    \left[\frac{k}{12}\right] & \text{if }k\not\equiv 2\pmod{12}.
  \end{cases}
\end{equation*}

\begin{conjecture}[Maeda~\cite{Maeda}]
  Let $m>1$ and 
  let $F$ be the characteristic polynomial of the Hecke operator $T_m$ acting
  on $S_k$.  Then $F$ is irreducible over $\QQ$ and the Galois group of its
  splitting field is the full symmetric group $\SS_d$, where $d$ is the
  dimension of $S_k$.
\end{conjecture}

TODO: Describe (relevant) previous results.

\section{The basic lemma and density estimates}

TODO: State and prove the basic lemma of Buzzard-Conrey-Farmer,
see~\cite{Buzzard} and~\cite[Lemma~4]{ConreyFarmer}.

TODO: State and prove density estimates for the three types of primes
occurring in the basic lemma.  These use the result of Frobenius on
splitting of primes, so we should state that as well.

\section{Implementation and results}
We implemented the basic approach described in the previous section
using the mathematical software Sage, see~\cite{Sage}.

Here is what we do for a fixed weight $k$:
\begin{enumerate}
  \item Compute the Victor Miller basis $\cB$ for $S_k$ up to precision
    $2(d+2)$, where $d$ is the dimension of $S_k$.
  \item Compute the matrix $M$ of the Hecke operator $T_2$ with respect to the
    basis $\cB$ -- this is very efficient since the basis $\cB$ is
    echelonized.
  \item\label{itm:random} Pick a random prime $p<2^{20}$.
  \item Reduce $M$ modulo $p$ and compute the characteristic polynomial
    $F_p\in \FF_p[X]$.
  \item Is $F_p$ irreducible?  If so, $p$ is a prime of type $I$.
  \item Factor $F_p$ over $\FF_p$ and use this factorization to decide whether
    $p$ is a prime of type $II$ or $III$.
  \item Repeat from step~(\ref{itm:random}) until we have found at
    least one prime of each type.
\end{enumerate}

TODO: find out which component of Sage is used in the main steps.

TODO: get timings for each main step for a large weight $k$.

TODO: run a consecutive version of this algorithm (going through primes in
order, from $p=2$ on) and see how it compares to our randomized approach.

\begin{theorem}[Maeda's conjecture holds for weights $\leq \bound$]
  Let $k\leq \bound$ and let $F$ be the characteristic polynomial of the
  Hecke operator $T_2$ acting on the space $S_k$ of cusp forms of weight
  $k$ and level $1$.  Then $F$ is irreducible over $\QQ$ and the Galois
  group of its splitting field is the full symmetric group $\SS_d$, 
  where $d$ is the dimension of the space $S_k$.
\end{theorem}


TODO: Describe the probability distribution we observe for the different types
of primes.

TODO: State some corollaries (look in the references for some nice
consequences).


\printbibliography

\end{document}


