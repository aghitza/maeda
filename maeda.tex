\documentclass[11pt]{article}

\usepackage[english]{babel}
\usepackage{csquotes}
\usepackage[style=alphabetic, 
            hyperref=auto,
            url=true,
            isbn=false,
            doi=false,
            block=none,
            maxnames=10,
            backend=biber]{biblatex}
\addbibresource{maeda.bib}
\renewcommand{\bibfont}{\normalfont\small}

\usepackage{amssymb}
\usepackage{amsthm}
\usepackage{amsmath}

% \usepackage[pagewise, mathlines]{lineno}
% \linenumbers
% \modulolinenumbers[2]

%\usepackage{gitinfo}
%\usepackage[english]{isodate}

\usepackage[a4paper, top=3cm, bottom=3cm]{geometry}
\usepackage[a4paper]{geometry}
\usepackage{numprint}
\usepackage[pdftex]{graphicx}

\theoremstyle{plain}
\newtheorem{theorem}{Theorem}[section]
\newtheorem{conjecture}[theorem]{Conjecture}
\newtheorem{lemma}[theorem]{Lemma}
\newtheorem{proposition}[theorem]{Proposition}
\newtheorem{corollary}[theorem]{Corollary}
\theoremstyle{definition}
\newtheorem{definition}[theorem]{Definition}
\theoremstyle{remark}
\newtheorem{remark}[theorem]{Remark}
\newtheorem{example}[theorem]{Example}

\numberwithin{equation}{section}
%\numberwithin{table}{section}
\renewcommand{\labelenumi}{(\arabic{enumi})}
\newcommand{\longto}{\longrightarrow}

\newcommand{\CC}{\mathbb{C}}
\newcommand{\ZZ}{\mathbb{Z}}
\newcommand{\FF}{\mathbb{F}}
\newcommand{\QQ}{\mathbb{Q}}
\newcommand{\NN}{\mathbb{N}}
\newcommand{\cB}{\mathcal{B}}
\newcommand{\cH}{\mathcal{H}}
\renewcommand{\SS}{\mathfrak{S}}
\renewcommand{\Im}{\operatorname{Im}}
\newcommand{\Frob}{\operatorname{Frob}}
\newcommand{\Gal}{\operatorname{Gal}}

\newcommand{\bound}{\numprint{10000}}

\newcommand{\GL}{\mathrm{GL}}
\newcommand{\SL}{\mathrm{SL}}

%\cleanlookdateon
\title{Experimental evidence for Maeda's conjecture on modular forms
%\footnote{Commit: \gitAbbrevHash{} on \printdate{\gitAuthorDate{}}}
}
\author{
Alexandru Ghitza\footnote{Research of the first author was supported by 
a Discovery Grant from the Australian Research Council.
Some of the computations described in this paper were performed on the Sage
cluster at the University of Washington, partly supported by National 
Science Foundation Grant No. DMS-0821725, held by William Stein.}  
{} and 
Angus McAndrew\\
Department of Mathematics and Statistics\\
University of Melbourne\\
{\tt aghitza@gmail.com}, {\tt mcandrew@student.unimelb.edu.au}
}
\date{}


\begin{document}
\thispagestyle{empty}

\maketitle
\begin{abstract}
  We describe a computational approach to the verification of Maeda's conjecture
  for the Hecke operator $T_2$ on the space of cusp forms of level 1. We provide
  experimental evidence for all weights less than $\bound$, as well as some
  applications of these results.
\end{abstract}


\section{Introduction}
\label{sect:introduction}
Modular forms come in many different types.  One of the most attractive
aspects of the theory is that, despite the apparent variety of definitions
and properties, there are some universal guiding principles (such as the
Langlands program) that serve to unify and motivate this diversity.  On the
other hand, there are some special properties that seem to occur in isolation.
One such instance is provided by a conjecture formulated by Maeda, which
indicates a behavior that seems to be specific to modular forms of level one
on $\GL_2$.  

Before describing Maeda's conjecture in more detail, we review some basic 
definitions and properties of modular forms.  For a thorough treatment of
the background needed in this paper, the reader is invited to
consult~\cite{Stein}.

Let $k\in\ZZ$.  A \emph{modular form} of level $1$ and weight $k$ is a
holomorphic function
\begin{equation*}
  f\colon\cH\longto\CC, \quad\text{where }
  \cH=\{z\in\CC\mid \Im z>0\},
\end{equation*}
satisfying
\begin{itemize}
  \item Modularity: for all $z\in\cH$ and all
    $g=\left(\begin{smallmatrix}a&b\\c&d\end{smallmatrix}\right)\in\SL_2(\ZZ)$,
      \begin{equation*}
        f\left(\frac{az+b}{cz+d}\right)=(cz+d)^kf(z).
      \end{equation*}
  \item Holomorphicity at $i\infty$: a holomorphic function $f$ satisfying the
    modularity condition satisfies $f(z+1)=f(z)$ for all $z\in\cH$, so it has
    a Fourier expansion
    \begin{equation*}
      f(z)=\sum_{n=-\infty}^\infty a_nq^n,\quad\text{where we set }
      q=e^{2\pi i z}.
    \end{equation*}
    We ask for $f$ to be \emph{holomorphic at $i\infty$}, i.e. that $a_n=0$
    for all $n<0$.
\end{itemize}

We say that a modular form $f$ is a \emph{cusp form} if $a_0=0$.  The cusp
forms of weight $k$ form a vector space $S_k$.  These vector spaces are
equipped with a family of \emph{Hecke operators} $T_m$ (for $m\in\NN$), whose
effect on the Fourier expansion $f(q)=\sum a_nq^n$ of $f\in S_k$ is given by
\begin{equation*}
  (T_m f)(q)=\sum_{n=1}^\infty \left(\sum_{d\mid\gcd(m,n)}d^{k-1}a_{mn/d^2}\right)q^n.
\end{equation*}

The complex vector space $S_k$ has dimension 
\begin{equation*}
  d=\begin{cases}
    \left[\frac{k}{12}\right]-1 & \text{if }k\equiv 2\pmod{12},\\
    \left[\frac{k}{12}\right] & \text{if }k\not\equiv 2\pmod{12}.
  \end{cases}
\end{equation*}

Let $F$ denote the characteristic polynomial of the operator $T_2$ acting on
$S_k$, and let $d=\dim S_k$.
In the 1970's, Yoshitaka Maeda noticed that $F$ is irreducible over $\QQ$ for
all $k$ such that $d\leq 12$.  In the 1990's, Lee-Hung~\cite{LeeHung} and
Buzzard~\cite{Buzzard} studied these polynomials further and observed in a
number of cases that the Galois group of $F$ is the symmetric group
$\SS_d$.  Shortly thereafter, Maeda made the following conjectural statement:

\begin{conjecture}[Maeda~\cite{HidaMaeda}]
  Let $m>1$ and 
  let $F$ be the characteristic polynomial of the Hecke operator $T_m$ acting
  on $S_k$.  Then 
  \begin{enumerate}
    \item the polynomial $F$ is irreducible over $\QQ$;
    \item the Galois group of the splitting field of $F$ is the full symmetric
      group $\SS_d$, where $d$ is the dimension of $S_k$.
  \end{enumerate}
\end{conjecture}

The conjecture has enjoyed constant attention over the last 15 years, with
theoretical as well as computational results.  We summarize the computational
verifications in Table~\ref{tbl:known}.

\begin{table}[h]
  \begin{center}
\begin{tabular}{l|r}
  Source & weights\\ \hline
  Lee-Hung~\cite{LeeHung} & $k\leq 62$, $k\neq 60$ \\
  Buzzard~\cite{Buzzard} & $k=12\ell$, $\ell$ prime, $2\leq\ell\leq 19$ \\
  Maeda~\cite{HidaMaeda} & $k\leq 468$ \\
  Conrey-Farmer~\cite{ConreyFarmer} & $k\leq 500$, $k\equiv 0\pmod 4$  \\
  Farmer-James~\cite{FarmerJames} & $k\leq \numprint{2000}$  \\
  Buzzard-Stein, Kleinerman~\cite{Kleinerman} & $k\leq\numprint{3000}$ \\
  Chu-Wee Lim~\cite{Lim} & $k\leq\numprint{6000}$ \\
  present paper & $k\leq\bound$  \\
\end{tabular}
\end{center}
\caption{Summary of known cases of Maeda's conjecture for $T_2$}
\label{tbl:known}
\end{table}

The theoretical results focus on whether the validity of the conjecture for
a given operator $T_m$ can be used to deduce the conjecture for other
operators $T_n$.  We state three such results, each giving a partial answer to
this question.

\begin{theorem}[Conrey-Farmer-Wallace~\cite{ConreyFarmerWallace}]
  \label{thm:CFW}
  Let $k$ be a positive even integer.  Suppose there exists $n\geq 2$ such
  that the operator $T_n$ acting on $S_k$ satisfies Maeda's conjecture.  Then
  so does $T_p$ acting on $S_k$, for every prime $p$ in the set of density
  $5/6$ defined by the conditions
  \begin{equation*}
    p\not\equiv \pm 1\pmod{5}\qquad\text{or}\qquad
    p\not\equiv \pm 1\pmod{7}.
  \end{equation*}
\end{theorem}

Stated differently, this says that if Maeda's conjecture in weight $k$ holds
for one index $n$, then the density of primes for which the conjecture
fails is at most $1/6$.  The next result considers only the irreducibility
part of the conjecture, but it is stronger since it says that the density of
primes for which the conjecture fails is zero.

\begin{theorem}[Baba-Murty~\cite{BabaMurty}]
  Let $k$ be a positive even integer.  Suppose there exists a prime $p$ such
  that the characteristic polynomial of $T_p$ acting on $S_k$ is irreducible
  over $\QQ$.  Then there exists $\delta>0$ such that
  \begin{equation*}
    \#\{\ell\leq N\text{ prime}\mid \text{charpoly}(T_\ell|S_k)\text{ is reducible}\}
    \ll \frac{N}{(\log N)^{1+\delta}}.
  \end{equation*}
\end{theorem}

Finally, Ahlgren proved a simple criterion for extending the validity of
Maeda's conjecture from one index to another, and used it together with some
computer work to prove the following result.

\begin{theorem}[Ahlgren~\cite{Ahlgren}]\label{thm:Ahl}
  Let $k$ be such that $d:=\dim S_k\geq 2$.  Suppose there exists $n\geq 2$
  such that the operator $T_n$ acting on $S_k$ satisfies Maeda's conjecture.
  Then
  \begin{enumerate}
    \item $T_p$ acting on $S_k$ satisfies Maeda's conjecture for all primes
      $p\leq\numprint{4000000}$;
    \item $T_n$ acting on $S_k$ satisfies Maeda's conjecture for all
      $n\leq\numprint{10000}$.
  \end{enumerate}
\end{theorem}


We can now state our main result.

\begin{theorem}\label{thm:main}
  Let $k\leq \bound$ and let
  \begin{align*}
    n\in &\{2,\ldots,\numprint{10000}\}\cup
    \{p\text{ prime}\mid 2\leq p\leq\numprint{4000000}\}\\
    &\cup\{p\text{ prime}\mid p\not\equiv\pm 1\pmod{5}\}\cup
    \{p\text{ prime}\mid p\not\equiv\pm 1\pmod{7}\}.
  \end{align*}
  Let $F$ be the characteristic polynomial of the
  Hecke operator $T_n$ acting on the space $S_k$ of cusp forms of weight
  $k$ and level $1$.  Then $F$ is irreducible over $\QQ$ and the Galois
  group of its splitting field is the full symmetric group $\SS_d$, 
  where $d$ is the dimension of the space $S_k$.
\end{theorem}
\begin{proof}
  The statement for $T_2$ is the result of the computations described below.  
  Given this, we deduce the result for the other $T_n$ by applying the results of
  Conrey-Farmer-Wallace and Ahlgren, as stated above.
\end{proof}

Our computational approach follows the ``multimodular'' method introduced by
Buzzard in~\cite{Buzzard} and refined by Conrey-Farmer in~\cite{ConreyFarmer}.
The main improvement is the use of random primes of moderate size, instead of
going through primes consecutively until suitable ones are found.  In
Section~\ref{sect:density} we describe the theoretical foundation of this
approach, and we estimate the densities of the different types of primes we
are looking for.  This provides us with expected running times for our
randomized algorithm, which we discuss in detail in
Section~\ref{sect:implementation}.  Finally, Section~\ref{sect:applications}
gives some direct corollaries of Theorem~\ref{thm:main} to some questions
about modular forms of level one.

\section{Polynomial factorization and Frobenius elements}
\label{sect:frobenius}
Since this correspondence plays a central role in the theoretical
underpinnings of our algorithm, we review some facts about factorization of
polynomials over finite fields and Frobenius elements in Galois groups.  These
ideas go back all the way to the beginnings of algebraic number theory,
appearing for instance in the work of Frobenius.  A fascinating exposition of the
mathematics and history of these ideas is given by Stevenhagen and Lenstra
in~\cite{StevenhagenLenstra}.

We start with a bit of terminology.  If $\tau$ is a permutation on $d$ letters, 
it can be decomposed into a product of disjoint cycles, uniquely up to
permutation of the cycles.  We say that $\tau$ has \emph{cycle pattern}
$d_1^{m_1}d_2^{m_2}\ldots d_t^{m_t}$ if its decomposition contains exactly
$m_j$ cycles of length $d_j$, for $j=1,\ldots,t$.  If $H$ is a polynomial in
$\FF_p[X]$, we say that $H$ has \emph{factorization pattern}
$d_1^{m_1}d_2^{m_2}\ldots d_t^{m_t}$ if $H$ has exactly $m_j$ irreducible
factors of degree $d_j$ over $\FF_p$.  We recall that $H$ is said to be
\emph{separable} if it has distinct roots over $\overline{\FF}_p$.

\begin{lemma}
  \label{lem:frobenius}
  Let $F\in\ZZ[X]$ be monic, let $p$ be a prime and let $F_p\in\FF_p[X]$ be
  the reduction of $F$ modulo $p$.  If $F_p$ is separable, then there exists
  an element $\tau$ of the Galois group of $F$ such that the cycle pattern of
  $\tau$ is the same as the factorization pattern of $F_p$.
\end{lemma}

We sketch a proof of this result.

Fix a prime $p$ and consider the field automorphism
$\sigma\colon\overline{\FF}_p\longto\overline{\FF}_p$ given by $\sigma(a)=a^p$.
Since $\sigma$ fixes the subfield $\FF_p$, it permutes the roots of any
polynomial $H\in\FF_p[X]$.  Moreover, Galois theory tells us that the cycle
pattern of $\sigma$ (viewed as a permutation) is the same as the factorization
pattern of $H$ over $\FF_p$. 

We now take a monic polynomial $F\in\ZZ[X]$ and we let $K/\QQ$ be its
splitting field, $\mathcal{O}_K$ the ring of integers of $K$, and $G$ the
Galois group of $K/\QQ$.  Let $\mathfrak{p}$ be a prime in $\mathcal{O}_K$
over $p$.  Suppose the reduction $F_p$ of $F$ modulo $p$ is a separable
polynomial (in this case, we say that $p$ is unramified in $K/\QQ$).  Then
there is a \emph{Frobenius element} $\Frob_{\mathfrak{p}}\in G$ determined
uniquely by the property
\begin{equation*}
  \Frob_{\mathfrak{p}}(\alpha)\equiv\sigma(\alpha)\pmod{\mathfrak{p}}\qquad
  \text{for all }\alpha\in\mathcal{O}_K.
\end{equation*}

This implies that $\Frob_{\mathfrak{p}}$ permutes the roots
$\alpha_1,\ldots,\alpha_d\in\mathcal{O}_K$ of $F$ in the exact same way as
$\sigma$ permutes the roots in $\overline{\FF}_p$ of $F_p$.  We conclude that
the cycle pattern $\Frob_{\mathfrak{p}}$ is the same as the factorization
pattern of $F_p$ over $\FF_p$.  Therefore we can take $\tau$ in the conclusion
of~\ref{lem:frobenius} to be $\Frob_{\mathfrak{p}}$.

Note that $\tau$ is not uniquely determined by $F$ and $p$, as the choice of a
prime $\mathfrak{p}$ of $\mathcal{O}_K$ above $p$ matters.  However, any two
such $\tau$ are conjugate in the Galois group.

The following result follows easily from Lemma~\ref{lem:frobenius} and the
fact that for any $F\in\ZZ[X]$ there are only finitely many primes $p$ (namely
the ones dividing the discriminant of $F$) for which $F_p$ is not separable.

\begin{theorem}[Frobenius]\label{thm:frobenius}
  Let $F\in\ZZ[X]$ be monic, let $K/\QQ$ be the splitting
  field of $F$ and let $G$ be the Galois group of $K/\QQ$.  
  Let $\deg F=m_1d_1+\ldots+m_td_t$ be a partition of $\deg F$.  
  The density of primes $p$ for which $F_p$ has factorization pattern
  $d_1^{m_1}\ldots d_t^{m_t}$ is equal to
  \begin{equation*}
    \frac{\#\{\sigma\in G\mid\text{the cycle pattern of $\sigma$ is 
    $d_1^{m_1}\ldots d_t^{m_t}$}\}
    }{\# G}.
  \end{equation*}
\end{theorem}


\section{The basic lemma and density estimates}
\label{sect:density}

Consider a monic polynomial $F\in\ZZ[X]$ of degree $d$.
Given a prime $p$, we denote by
$F_p\in\FF_p[X]$ the reduction modulo $p$ of $F$.  We say that the prime
$p$ is
\begin{enumerate}
  \item \emph{of type I} if $F_p$ is irreducible over $\FF_p$;
  \item \emph{of type II} if $F_p$ factors over $\FF_p$ into a product of
    distinct irreducible factors
    \begin{equation*}
      F_p=f_0f_1\cdots f_t
    \end{equation*}
    with
    \begin{align*}
      \deg f_0 &= 2\\
      \deg f_j &\text{ odd for }j=1,\ldots,t;
    \end{align*}
  \item \emph{of type III} if $F_p$ factors over $\FF_p$ into a product of
    distinct irreducible factors
    \begin{equation*}
      F_p=f_0f_1\cdots f_t
    \end{equation*}
    with $\deg f_0>d/2$ and prime.
\end{enumerate}

\begin{remark}\label{rmk:type4}
  Hida and Maeda use a similar approach, but replace primes of type III with
  \emph{primes of type IV}, i.e. $p$ such that $F_p=f_0f_1$ with $f_0$, $f_1$
  distinct and irreducible, and $\deg f_0=1$.  We will see below that primes
  of type III are significantly more common (and therefore better suited for
  our algorithm) than those of type IV.
\end{remark}

\begin{remark}
These types are not necessarily mutually exclusive: if $d$ itself is
prime, then a prime $p$ of type I is clearly also of type III.
\end{remark}

\begin{remark}
In either of the three types, the conditions imply that the reduced polynomial
$F_p$ is separable (i.e. it has distinct roots over $\overline{\FF}_p$).

  If there exists a prime of type I, then $F$ is irreducible. Since $\QQ$ is a field of
  characteristic 0, and $F \in \QQ[X]$, this implies that $F$ is separable. Let $p$ and 
  $q$ be primes of types II and III, respectively. Then $F_p = f_0 f_1 \ldots f_{t_1}$
  and $F_q = g_0 g_1 \ldots g_{t_2}$, with $\deg f_0 = 2$ and $\deg g_0 > d/2$ is prime.
  Thus the roots of $F_p$ and $F_q$ are distinct, since any of the $f_i$ or $g_i$ has
  distinct roots (by the above argument for the prime of type I), and if any two factors
  share a root then either their product is reducible or the factors are equal to one
  another. Since this contradicts our assumption on the factorisations of $F_p$ and
  $F_q$, they must have distinct roots.
\end{remark}

Our computational approach to Maeda's conjecture is based on the following
result, first proved in a special case in~\cite{Buzzard} and then generalized
in~\cite{ConreyFarmer}.

\begin{lemma}[Buzzard-Conrey-Farmer]
  Let $F\in\ZZ[X]$ be a monic polynomial of degree $d$.  Suppose that $F$ has
  primes $\ell$, $p$ and $q$ of respective types I, II and III.  Then $F$
  is irreducible over $\QQ$ and its splitting field over $\QQ$ has full Galois
  group $\SS_d$.
\end{lemma}
\begin{proof}
  TODO: reread this!

  The fact that $F$ is irreducible is immediate from the existence of a prime of type
  I. 
  Let $K/\QQ$ be the splitting field of $F$ and let $G$ be the Galois group of
  $K/\QQ$.  Since $F$ is irreducible, $G$ is a transitive subgroup of $\SS_d$. 

  So $\Frob_{\mathfrak{p}}$ has cycle type $2,\deg(f_1),\ldots,\deg(f_{t_1})$ and
  $\Frob_{\mathfrak{q}}$ has cycle type $p,\deg(g_1),\ldots,\deg(g_{t_2})$, where 
  $p>d/2$ is prime. Thus for $x_1=\prod_{i=1}^{t_1}\deg f_i$ and $x_2=\prod_{i=1}^{t_2}\deg g_i$,
  we have that $(\Frob_{\mathfrak{q}})^{x_1}$ is a transposition and 
  $(\Frob_{\mathfrak{r}})^{x_2}$
  is a cycle of length $p$. The fact that these elements are in $G$ implies $G=\SS_d$ as follows:
  Put an equivalence relation $\sim$ on $S=\{1,2,\ldots,d\}$ by $a\sim b$ if $a=b$ or if the
  transposition $(a~b)$ is in $G$. Since $G$ is
  transitive, each equivalence class has the same number of elements, denote this
  $n$, and it follows that $n|d$. Note that $n>1$ since $G$ contains at least one
  transposition. Let $T=\{e_1,e_2,\ldots,e_p\}$ be the subset of $S$ permuted by
  ($\Frob_{\mathfrak{r}})^{x_2}$, and let $G_T$ be the subgroup of $G$ fixing $S\setminus T$.
  Define an equivalence relation on $T$ by $a\simeq b$ if $a=b$ or if the transposition
  $(a~b) \in G_T$. As before, each equivalence class has the same number of elements,
  say $m$, and $m\mid p$. Since $n>1$, we have $m>1$, so $m=p$ since $p$ is prime.
  But $n\geq m$ because $G_T\subset G$. Thus $n>d/2$, so $n=d$. This implies $G=\SS_d$. 
\end{proof}

Our algorithm will consist of picking random primes and checking whether they
are of type I, II or III for the characteristic polynomial of the Hecke
operator $T_2$.  According to Theorem~\ref{thm:frobenius}, 
it is therefore important to estimate the number of permutations having
certain types of cycle patterns.  For a fixed pattern, the following
well-known result 
(see, for instance, Proposition 1.3.2 of~\cite{Stanley1}) gives an exact
expression for the number of permutations.

\begin{lemma}\label{lem:cycletype}
  Let an element $\sigma$ of $\SS_d$ have cycle type $d_1^{m_1}d_2^{m_2}\ldots d_t^{m_t}$, 
  where $m_i$ is the number of times a cycle of length $d_i$ appears in the cycle
  decomposition of $\sigma$. (Note: $m_1d_1+m_2d_2+\ldots +m_td_t=d$.) The number
   of elements of $\SS_d$ of cycle type $d_1^{m_1}d_2^{m_2}\ldots d_t^{m_t}$ is equal to
  \begin{equation*}
    C(d_1^{m_1}d_2^{m_2}\ldots d_t^{m_t})=\frac{d!}{\prod_{j=1}^t\left(d_j^{m_j}m_j!\right)}.
  \end{equation*}
\end{lemma}


\begin{proposition}\label{prop:type1}
  The density of primes of type I is 
  \begin{equation*}
    D_I(d)=\frac{1}{d}.
  \end{equation*}
\end{proposition}
\begin{proof}
  Primes of type I correspond to $d$-cycles in $\SS_d$.  Each such cycle can
  be written uniquely as a sequence $1,a_1,\ldots,a_{d-1}$, where
  $a_1,\ldots,a_{d-1}\in\{2,\ldots,d\}$ can appear in any order.  Therefore
  there are $(d-1)!$ $d$-cycles, and by Theorem~\ref{thm:frobenius},
  the density of primes of type I is
  \begin{equation*}
    \frac{(d-1)!}{d!}=\frac{1}{d}.
  \end{equation*}
\end{proof}

\begin{proposition}\label{prop:type2}
  The density of primes of type II is given by\footnote{Recall that the \emph{double factorial} is...}
  \begin{align*}
    D_{II}(1)&=1,\\
    D_{II}(d)&=\begin{cases}
        \frac{1}{2(d-2)!}\left((d-3)!!\right)^2,&\text{if }d\text{ is even}\\
        \frac{1}{2(d-2)!}\left((d-4)!!\right)^2(d-2),&\text{if }d\text{ is odd and }d>1
    \end{cases}
  \end{align*}
  For $d>2$, $D_{II}(d)$ satisfies
  \begin{equation*}
    D_{II}(d)>\frac{3}{4\sqrt{2\pi}\frac{1}{\sqrt{d}}
  \end{equation*}
\end{proposition}
\begin{proof}
  Primes of type II correspond to elements in $\SS_d$ containing a $2$-cycle and no other
  even cycles. There are $\binom{d}{2}$ $2$-cycles in $\SS_d$; fixing a $2$-cycle, we 
  need the number of elements of odd order in $\SS_{d-2}$. Denote this by $\sigma(d-2)$. Then
  \begin{align*}
    D_{II}(d)&=\frac{1}{d!}\binom{d}{2}\sigma(d-2)\\
             &=\frac{\sigma(d-2)}{2(d-2)!}
  \end{align*}
  From~\cite{Riordan} we have the following:
  \begin{equation*}
    \sigma(n)=\begin{cases}
        (n-1)!!,&\text{if }n\text{ is even}\\
        (n-2)!!n,&\text{if }n\text{ is odd.}
      \end{cases}
  \end{equation*}
  Thus we have $D_{II}(d)$ as above. We turn to the inequality.

  Let $d=2c$ if $d$ is even and $d=2c+1$ if $d$ is odd, for some $c \in \ZZ$. Then $\sigma(d-2)$
  can be expanded as follows:
  \begin{equation*}
    \sigma(d-2)=\begin{cases}
      \left[\frac{(2c-3)!}{2^{c-2}(c-2)!}\right]^2,&\text{if }d=2c\\
      \left[\frac{(2c-3)!}{2^{c-2}(c-2)!}\right]^2(2c-1),&\text{if }d=2c+1
    \end{cases}
  \end{equation*}
  We now obtain the inequality given in the statement as follows:
  \begin{align*}
    D_{II}(d)&=\frac{\sigma(d-2)}{2(d-2)!}\\
             &=\begin{cases}
                  \left[\frac{(2c-3)!}{(c-2)!}\right]^2\frac{1}{2^{2c-4}}\frac{1}{2(2c-2)!},&\text{if }d=2c\\
                  \left[\frac{(2c-3)!}{(c-2)!}\right]^2\frac{2c-1}{2^{2c-4}}\frac{1}{2(2c-1)!},&\text{if }d=2c+1
                 \end{cases}\\
             &=\frac{1}{2^{2c-3}(2c-2)}\frac{(2c-3)!}{[(c-2)!]^2}
  \end{align*}
  We make use of the following inequalities from [TODO: reference]:
  \begin{equation*}
    \sqrt{2\pi}n^{n+\frac{1}{2}}e^{-n+\frac{1}{12n+1}}<n!<\sqrt{2\pi}n^{n+\frac{1}{2}}e^{-n+\frac{1}{12n}}
  \end{equation*}
  Then by using the lower bound for the numerator and the upper bound for the denominator,
  we see the following:
  \begin{align*}
    D_{II}(d)&>\frac{2c-3}{2c-2}\frac{1}{2\sqrt{\pi(c-2)}}e^{\frac{1}{24(c-2)+1}-\frac{1}{6c-12}}\\
             &>\frac{3/(8\sqrt{\pi}}{\sqrt{c-2}}\\
             &>\frac{3}{4\sqrt{2\pi}}\frac{1}{\sqrt{d}}
  \end{align*}
  Where the first inequality comes from the following:
  \begin{equation*}
    e^{\frac{1}{24(c-2)+1}-\frac{1}{6c-12}}<1\text{ if }c\geq1,\text{ and }\frac{2c-3}{2c-2}>3/4\text{ if }c\geq2
  \end{equation*}
\end{proof}



\begin{proposition}
  The density of primes of type III is
  \begin{equation*}
    D_{III}(d)=\sum_{d/2<\ell\leq d, \,\ell\text{ prime}} \frac{1}{\ell}.
  \end{equation*}
  If $d>2$, then
  \begin{equation*}
    D_{III}(d)>\frac{1}{d}.
  \end{equation*}
\end{proposition}
\begin{proof}
  Fix a prime $\ell$ such that $d/2<\ell\leq d$.  According to
  Theorem~\ref{thm:frobenius}, we need to count the number of elements of
  $\SS_d$ that contain an $\ell$-cycle.  Choosing the $\ell$-cycle itself
  involves the $\binom{d}{\ell}$ ways of picking its constituents, which can
  then be rearranged within the cycle in $(\ell-1)!$ ways.  It remains to
  take into account the number of permutations of the remaining $d-\ell$ symbols,
  so overall we have
  \begin{equation*}
    \binom{d}{\ell}(\ell-1)!(d-\ell)!=\frac{d!}{\ell}
  \end{equation*}
  elements of $\SS_d$ containing an $\ell$-cycle, which gives the stated
  density.

  The inequality given in the statement follows from Bertrand's postulate
  (proved by Chebyshev), which says that for any integer $n>1$ there is at
  least one prime $\ell$ such that $n<\ell<2n$.
\end{proof}

We can get a much better lower bound on the density $D_{III}$ by using some
recent results of Dusart on explicit estimates for sums over primes.

\begin{theorem}[Dusart, Theorem 6.10 in~\cite{Dusart}]
  Let $B\approx 0.26149$ denote the Meissel-Mertens constant.  For
  all $x>1$ we have
  \begin{equation}\label{eq:rec_lower}
    \log\log x+B-\left(\frac{1}{10\log^2 x}+\frac{4}{15\log^3 x}\right)\leq
    \sum_{p\leq x}\frac{1}{p}.
  \end{equation}
\end{theorem}

We will also need an upper bound on the sum of the reciprocals of primes up to
$x$, but Dusart's upper bound only holds for $x\geq\numprint{10372}$.  For our
purposes, the following weaker result is sufficient: for all $x>1$ we have
\begin{equation}\label{eq:rec_upper}
  \sum_{p\leq x}\frac{1}{p}\leq\log\log x + B +\frac{1}{\log^2 x}.
\end{equation}
(This inequality can be found in Theorem 8.8.5 of~\cite{BachShallit}.)

\begin{proposition}\label{prop:type3}
  If $d>10$, then
  \begin{equation*}
    D_{III}(d)>\frac{1}{3\log d}.
  \end{equation*}
\end{proposition}
\begin{proof}
  We put together inequalities~\eqref{eq:rec_lower} and~\eqref{eq:rec_upper}
  to get
  \begin{equation*}
    D_{III}(d)>\log\log d-\log\log\frac{d}{2}-\frac{1}{10\log^2 d}
    -\frac{4}{15\log^3 d}-\frac{1}{\log^2\frac{d}{2}}.
  \end{equation*}
  We write
  \begin{equation*}
    \log\log d - \log\log\frac{d}{2} =\log\left(1+\frac{\log 2}{\log d-\log
    2}\right)
  \end{equation*}
  and use the inequality
  \begin{equation*}
    \log(1+x)\geq x-\frac{x^2}{2}+\frac{x^3}{3}-\frac{x^4}{4}
    \qquad\text{for all }1<x\leq 1
  \end{equation*}
  to get that for all $d\geq 4$
  \begin{align*}
    D_{III}(d)>&\frac{1}{\log d-\log 2}\left[
      \log 2
      -\left(\frac{\log^2 2}{2}+\frac{11}{10}\right)\frac{1}{\log d-\log
      2}\right.\\
      &\left.\phantom{\frac{1}{\log d-\log 2}} 
      -\left(\frac{4}{15}-\frac{\log^3 2}{3}\right)\frac{1}{(\log d-\log 2)^2}
      -\frac{\log^4 2}{4}\frac{1}{(\log d-\log 2)^3}
      \right]\\
      >&\frac{1}{\log d}\left[
      0.693-1.341\frac{1}{\log d-\log 2}
      -0.156\frac{1}{(\log d-\log 2)^2}\right.\\
      &\left.\phantom{\frac{1}{\log d}}
      -0.058\frac{1}{(\log d-\log 2)^3}
      \right].
  \end{align*}
  If $d>94$, then the expression in the brackets is bigger than $1/3$, and we
  get the desired inequality.  We check that it holds for the remaining cases
  $10<d\leq 94$ by computation.
\end{proof}

For completeness, we treat the case of primes of type IV, as defined in
Remark~\ref{rmk:type4}.

\begin{proposition}\label{prop:type4}
  The density of primes of type IV is
  \begin{equation*}
    D_{IV}(d)=\frac{1}{d-1}.
  \end{equation*}
\end{proposition}
\begin{proof}
  We need to count the number of $(d-1)$-cycles in $\SS_d$.  There are $d$
  choices for the letter that is fixed, and $(d-2)!$ choices for permuting the
  other letters appropriately, therefore the density of primes of type IV is
  \begin{equation*}
    \frac{d(d-2)!}{d!}=\frac{1}{d-1}.
  \end{equation*}
\end{proof}


\section{Implementation and results}
\label{sect:implementation}
Our approach is a randomized version of the algorithm
from~\cite{ConreyFarmer}, based on the results introduced in the previous
section.  We implemented this algorithm using the mathematical software Sage, 
see~\cite{Sage}.

Here is what we do for a fixed weight $k$:
\begin{enumerate}
  \item\label{itm:vmbasis} Compute the Victor Miller basis $\cB$ for $S_k$ up to precision
    $2(d+2)$, where $d$ is the dimension of $S_k$.
  \item\label{itm:hecke} Compute the matrix $M$ of the Hecke operator $T_2$ with respect to the
    basis $\cB$ -- this is very efficient since the basis $\cB$ is
    echelonized.
  \item\label{itm:random} Pick a random prime $p<2^{20}$, uniformly over this
    range.  (This choice of upper
    bound gives a large enough range so that it is likely to contain primes of
    type we are looking for, but not so large that the arithmetic over $\FF_p$
    gets too expensive.)
  \item Reduce $M$ modulo $p$ and compute the characteristic polynomial
    $F_p\in \FF_p[X]$.
  \item Is $F_p$ irreducible?  If so, $p$ is a prime of type I.
  \item Factor $F_p$ over $\FF_p$ and use this factorization to decide whether
    $p$ is a prime of type II or III.
  \item Repeat from step~(\ref{itm:random}) until we have found at
    least one prime of each type.
\end{enumerate}

According to Propositions~\ref{prop:type1}, \ref{prop:type2} and
\ref{prop:type3}, we expect to look on average at $d$ primes before we find
one of type I, at $\sqrt{d}$ primes to find one of type II, and at $3\log d$
primes to find one of type III.

The actual performance of this algorithm (as well as a comparison to the
consecutive version of the algorithm, used in~\cite{ConreyFarmer}) is
illustrated in Figure~\ref{fig:histogram}.  Some care needs to be taken in
interpreting the graphs:
\begin{itemize}
  \item There is no difference in running times for Steps~(\ref{itm:vmbasis}) 
    and~(\ref{itm:hecke}), which 
    are common between the two algorithms.
  \item As the weight increases, the major component of the running time is
    finding a prime of type I.  Therefore, even though the randomized
    algorithm does much better at finding primes of types II and III, this
    advantage has only a minor impact on the overall running time.
  \item In the range illustrated in the graphs (i.e. weights less than
    $\numprint{2000}$), the randomized algorithm required on average one third
    of the number of primes needed by the consecutive algorithm.  However,
    some of this is counteracted by the fact that the consecutive algorithm
    works with much smaller primes, which are faster to test.
\end{itemize}

Overall, for weights less than $\numprint{2000}$, the randomized algorithm was
about twice as fast as the consecutive one.  

TODO: find out which component of Sage is used in the main steps.

TODO: get timings for each main step for a large weight $k$.



\begin{figure}[h]
  \begin{center}
  \raisebox{2.5cm}{
  \begin{tabular}{lrr}
    & consec & random\\
    min & $1.85$ & $0.01$\\
    max & $10.00$ & $6.50$\\
    med & $2.93$ & $0.69$\\
    mean & $3.21$ & $0.96$
  \end{tabular}}\qquad
  \includegraphics[width=3.6in, height=2.2in]{type1.png}

  \raisebox{2.5cm}{
  \begin{tabular}{lrr}
    & consec & random\\
    min & $1.85$ & $0.01$\\
    max & $10.00$ & $6.50$\\
    med & $2.93$ & $0.69$\\
    mean & $3.21$ & $0.96$
  \end{tabular}}\qquad
\includegraphics[width=3.6in, height=2.2in]{type2.png}

  \raisebox{2.5cm}{
  \begin{tabular}{lrr}
    & consec & random\\
    min & $1.85$ & $0.01$\\
    max & $10.00$ & $6.50$\\
    med & $2.93$ & $0.69$\\
    mean & $3.21$ & $0.96$
  \end{tabular}}\qquad
\includegraphics[width=3.6in, height=2.2in]{type3.png}
\end{center}
\caption{Histograms illustrating the number of primes tested before finding a
  prime of type I, II, respectively III, in weights up to $\numprint{2000}$.  
  In each graph, the numbers on the
$x$-axis represent the ratio $N/E$ of the actual number of primes tested over
the expected number of primes (coming from the densities described in
Section~\ref{sect:density}).  The $y$-value represents the number of weights
featuring (a small neighborhood of) that particular ratio $N/E$.  The blue 
continuous line corresponds to our randomized algorithm, while the red dotted 
line corresponds to the consecutive algorithm from~\cite{ConreyFarmer}.  
As an example: in the top graph, the global maximum on the continuous line is
at $(0.1, 101)$, meaning that for $101$ weights, the number of candidates for
a prime of type I tested in the randomized algorithm was about $1/10$ of the 
expected number of primes.}
\label{fig:histogram}
\end{figure}


\section{Some applications}
\label{sect:applications}

We record some immediate consequences of Theorem~\ref{thm:main}.

\subsection{Non-vanishing of $L$-functions}
Recall that the \emph{$L$-function} associated to an eigenform 
$f=\sum_{n=1}^\infty a_n q^n$ of weight $k$ is given by
\begin{equation*}
  L(f, s)=\sum_{n=1}^\infty \frac{a_n}{n^s}.
\end{equation*}
If $k\equiv 2\pmod{4}$, the functional equation of $L$ implies that
$L(f, k/2)=0$.  It is believed that if $k\equiv 0\pmod{4}$, then $L(f, k/2)\neq
0$.  The following result follows immediately from work of Conrey-Farmer: 

\begin{corollary}[see Theorem 1 in~\cite{ConreyFarmer}]
  Suppose $k\equiv 0\pmod{4}$ and $k\leq\bound$.  Then
  $L(f, k/2)\neq 0$ for any cuspidal eigenform $f$ of level $1$ and weight $k$.
\end{corollary}

\subsection{Base change for totally real fields}
It is in the context of this work of Hida and Maeda that
Maeda's conjecture was formulated.  We content ourselves with giving a general
description of this application, and we refer the interested reader
to~\cite{HidaMaeda} for details. 

Let $f\in S_k$ be a Hecke eigenform.  For each prime $p$, there is a $p$-adic
Galois representation
\begin{equation*}
  \rho\colon\Gal\left(\overline{\QQ}/\QQ\right)
  \longto\GL_2\left(\overline{\QQ}_p\right).
\end{equation*}
There is an Artin $L$-function $L(\rho, s)$ attached to $\rho$, and the
relation between $\rho$ and $f$ can be summarized by
\begin{equation*}
  L(\rho, s) = L(f, s).
\end{equation*}

Now let $E$ be a number field.  There is a purely algebraic notion of a
cohomological eigenform $\hat{f}$ on $\GL_2(\mathbb{A}_E)$, where 
$\mathbb{A}_E$ is the ring
of adeles of $E$.  We say that $\hat{f}$ is a \emph{base change of $f$ to $E$}
if
\begin{equation*}
  L(\hat{f}, s) = L(\rho_E, s),
\end{equation*}
where $\rho_E\colon\Gal(\overline{\QQ}/E)\longto\GL_2(\overline{\QQ}_p)$ is
the restriction of $\rho$ to $E$.

The work of Hida and Maeda, together with Theorem~\ref{thm:main}, implies that
for $k\leq$ TODO and a totally real field $E$ satisfying some ramification
conditions, any eigenform $f\in S_k$ has a base change to $E$.

\subsection{Eigenforms divisible by eigenforms}
It is easy to see from the definition of a modular form that if $f_1$ and
$f_2$ are modular forms of respective weights $k_1$ and $k_2$, then the
product $f_1f_2$ is a modular form of weight $k_1+k_2$.  In other words,
modular forms of all weights put together form a graded algebra
\begin{equation*}
  M=\bigoplus_{k\in\ZZ} M_k.
\end{equation*}

A natural question is whether the product of eigenforms can be an eigenform.
This will clearly happen for small weights (for instance, when the product
lives in a one-dimensional space of cusp forms).  Since the Hecke operators
do not act on the entire algebra $M$ of modular forms (they act
differently on the graded pieces $M_k$), it seems reasonable that the
one-dimensional coincidences are the only situation in which a product of
eigenforms is an eigenform.  Such questions have been studied by several
authors, with the latest results appearing in a recent paper by
Beyerl-James-Xue~\cite{BeyerlJamesXue}.  They consider the more general question
of divisibility of an
eigenform by another eigenform, i.e. relations of the form $h=fg$ where
$f,g,h$ are modular forms and $f,h$ are eigenforms.  The relation with Maeda's 
conjecture is discussed in Section 6 of~\cite{BeyerlJamesXue}, and 
Theorem~\ref{thm:main} implies the following result.

\begin{corollary}
  Let $h$ be a cuspidal eigenform of weight $\leq\bound$, and let 
  $f\in M_l$ be an eigenform
  (which could be cuspidal or Eisenstein).  Then $h=fg$ for some modular form
  $g\in M_k$ with $k>2$ if and only if we are in one of the cases listed in 
  Table~\ref{tbl:div_eigen}.
\end{corollary}

\begin{table}[h]
  \begin{center}
  \begin{tabular}{r|r|l}
    weight of $f$ & weight of $g$ modulo $12$ & nature of $f$ \\ \hline
    $4$ & $0,4,6,10$ & Eisenstein \\
    $6$ & $0,4,8$ & Eisenstein \\
    $8$ & $0,6$ & Eisenstein \\
    $10$ & $0,4$ & Eisenstein \\
    $12$ & $0,2,4,6,8,10$ & cuspidal\\
    $14$ & $0$ & Eisenstein \\
    $16$ & $0,4,6,10$ & cuspidal\\
    $18$ & $0,4,8$ & cuspidal\\
    $20$ & $0,6$ & cuspidal\\
    $22$ & $0,4$ & cuspidal\\
    $26$ & $0$ & cuspidal
  \end{tabular}
\end{center}
  \caption{The only cases in which a cuspidal eigenform of weight $\leq\bound$
  can be factored into $h=fg$ with $f$ an eigenform.}
  \label{tbl:div_eigen}
\end{table}

\subsection{Distinguishing Hecke eigenforms}

How many initial Fourier coefficients are necessary to completely determine a
Hecke eigenform?  Theorem 1 in~\cite{Ghitza} says that $a_2$, $a_3$ and $a_4$
are sufficient, but our computational verification of Maeda's conjecture gives
a stronger result:

\begin{corollary}[see Theorem 6 in~\cite{Ghitza}]
  Comparing the Fourier coefficient $a_2$ is sufficient to distinguish any two
  cuspidal eigenforms of level $1$ and weights $\leq\numprint{10000}$.
\end{corollary}



\printbibliography

\end{document}


