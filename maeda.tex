\documentclass[11pt]{article}

\usepackage[english]{babel}
\usepackage{csquotes}
\usepackage[style=numeric, 
            hyperref=auto,
            url=true,
            isbn=false,
            doi=false,
            block=none,
            maxnames=10,
            backend=biber]{biblatex}
\addbibresource{maeda.bib}
\renewcommand{\bibfont}{\normalfont\small}

\usepackage{amssymb}
\usepackage{amsthm}
\usepackage{amsmath}

\usepackage[pagewise, mathlines]{lineno}
\linenumbers
\modulolinenumbers[2]

\usepackage{gitinfo}
\usepackage[english]{isodate}

\theoremstyle{plain}
\newtheorem{theorem}{Theorem}[section]
\newtheorem{conjecture}[theorem]{Conjecture}
\newtheorem{proposition}[theorem]{Proposition}
\theoremstyle{definition}
\newtheorem{definition}[theorem]{Definition}
\theoremstyle{remark}
\newtheorem{remark}[theorem]{Remark}
\newtheorem{example}[theorem]{Example}

\numberwithin{equation}{section}
\numberwithin{table}{section}
\renewcommand{\labelenumi}{(\alph{enumi})}

\newcommand{\CC}{\mathbb C}
\newcommand{\ZZ}{\mathbb Z}
\newcommand{\FF}{\mathbb F}
\newcommand{\QQ}{\mathbb Q}
\newcommand{\HH}{\mathbb H}
\renewcommand{\SS}{\mathfrak S}

\newcommand{\bound}{10\,000}

\newcommand{\GL}{{\rm GL}}
\newcommand{\SL}{{\rm SL}}

\title{Experimental evidence for Maeda's conjecture on modular forms}
\author{
Alexandru Ghitza\footnote{Research of the first author was supported by 
a Discovery Grant from the Australian Research Council.}  
{} and 
Angus McAndrew\\
Department of Mathematics and Statistics\\
University of Melbourne\\
{\tt aghitza@gmail.com}, {\tt mcandrew@student.unimelb.edu.au}
}
\date{}


\begin{document}
\thispagestyle{empty}

\maketitle
\begin{abstract}
  We describe a computational verification of Maeda's conjecture for
  all weights less than $\bound$.
\end{abstract}


\section{Introduction}

Motivation.

Define modular forms of level 1 and Hecke operators.

\begin{conjecture}[Maeda~\cite{Maeda}]
\end{conjecture}

Describe (relevant) previous results.

\section{The basic lemma and density estimates}

State and prove the basic lemma of Buzzard-Conrey-Farmer,
see~\cite{Buzzard} and~\cite[Lemma~4]{ConreyFarmer}.

State and prove density estimates for the three types of primes
occurring in the basic lemma.  These use the result of Frobenius on
splitting of primes, so we should state that as well.

\section{Implementation and results}
We implemented the basic approach described in the previous section
using the mathematical software Sage~\cite{Sage}.

Describe implementation.

\begin{theorem}[Maeda's conjecture holds for weights $\leq \bound$]
  Let $k\leq \bound$ and let $F$ be the characteristic polynomial of the
  Hecke operator $T_2$ acting on the space $S_k$ of cusp forms of weight
  $k$ and level $1$.  Then $F$ is irreducible over $\QQ$ and the Galois
  group of its splitting field is the full symmetric group $\SS_d$, 
  where $d$ is the dimension of the space $S_k$.
\end{theorem}


Describe the probability distribution we observe for the different types
of primes.

State some corollaries (look in the references for some nice
consequences).


\printbibliography

\end{document}


